\section{Performance Analysis}

In this section, we evaluate the performance of the proposed convolution co-processor. Metrics include processing throughput, latency, resource utilization, and energy efficiency. Comparisons are made with a reference CPU-only implementation on the ZYNQ7000 processing system.

\subsection{Experimental Setup}
The experiments were performed on a Digilent ZedBoard development board with the following specifications:

\begin{itemize}[noitemsep]
  \item Processing System: Dual-core ARM Cortex-A9, 667 MHz
  \item FPGA: XC7Z020 (Artix-7), 53k LUTs, 106k FFs, 4.9 Mb BRAM
  \item Clock frequency of co-processor: \SI{100}{\MHz}
  \item Test images: resolution $640 \times 480$, 32-bit RGBA
  \item Convolution kernel: $3\times3$
\end{itemize}

\subsection{Latency and Throughput}

The latency $T_\text{latency}$ of the co-processor is measured as the time between issuing a convolution request and receiving the processed data:
\begin{equation}
T_\text{latency} = T_\text{transfer} + T_\text{compute} + T_\text{response}
\end{equation}

Throughput $R_\text{throughput}$ is calculated as:
\begin{equation}
R_\text{throughput} = \frac{\text{Number of pixels processed}}{T_\text{latency}}
\end{equation}

\begin{table}[H]
\centering
\caption{Latency and throughput for processing new versus in-memory images}
\begin{tabular}{c c c}
\hline
In memory & Latency [ms] & Throughput [MPix/s] \\
\hline
No & -- & -- \\
Yes & -- & -- \\
\hline
\end{tabular}
\label{tab:latency-throughput}
\end{table}

\subsection{Resource Utilization}

The FPGA resource usage of the convolution co-processor is summarized in Table~\ref{tab:resource-util}:

\begin{table}[H]
\centering
\caption{FPGA Resource Utilization}
\begin{tabular}{l c c}
\hline
Resource & Used & Available \\
\hline
LUTs & -- & -- \\
Flip-Flops & -- & -- \\
BRAM [Kb] & -- & -- \\
DSP Slices & -- & -- \\
\hline
\end{tabular}
\label{tab:resource-util}
\end{table}

\subsection{Comparison with CPU Implementation}

For reference, a CPU-only implementation, as a FreeRTOS task with highest priority, was run on the ARM Cortex-A9 core. Table~\ref{tab:cpu-vs-fpga} summarizes the speed-up achieved:

\begin{table}[H]
\centering
\caption{Speed-Up of FPGA Co-Processor vs CPU}
\begin{tabular}{c c}
\hline
CPU Latency [ms] & FPGA Speed-Up \\
\hline
-- & -- \\
\hline
\end{tabular}
\label{tab:cpu-vs-fpga}
\end{table}

% If possible and if we have time enough
\subsection{Energy Efficiency}

Energy consumption was measured for the convolution co-processor using onboard power monitoring or external measurement tools. The energy efficiency $\eta$ is defined as the number of pixels processed per joule of energy consumed:

\begin{equation}
\eta = \frac{\text{Number of pixels processed}}{E_\text{total}} \quad [\text{MPixels/J}]
\end{equation}

where $E_\text{total}$ is the total energy consumed during the convolution operation.

\begin{table}[H]
\centering
\caption{Energy efficiency of the co-processor for CPU and FPGA}
\begin{tabular}{c c c}
\hline
Platform & Energy [mJ] & Efficiency [MPix/J] \\
\hline
$FPGA$ & -- & -- \\
$CPU$ & -- & -- \\
\hline
\end{tabular}
\label{tab:energy-efficiency}
\end{table}